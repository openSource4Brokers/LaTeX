\documentclass[10pt,a4paper]{article}
\usepackage[utf8]{inputenc}
\usepackage[english]{babel}
\usepackage{amsmath}
\usepackage{amsfonts}
\usepackage{amssymb}
\usepackage[left=2cm,right=2cm,top=2cm,bottom=2cm]{geometry}

\author{Arun Prasaad Gunasekaran}
\title{Document types, title contents and environments}
%\date{31 July 2015}
\date{\today}
%\institution{IISc} %Beamer command

\begin{document}
\maketitle

Documents are of many types, and they are defined by the class files. So, the default class is article.

Other documents are reports, chapters, letters, beamer, papers, proposals, etc.,

You have facility to create your own document class using the class files having the extension .cls

The default standard fonts in LaTeX are 10, 11, 12. However, short-range increase/decrease of fonts is possible.

The user details are: author name, date, title, sub-title, institution, logo etc.,

All of these come under the make title command. All these fields are not available for all class files. They vary for different class files.

Commands and Environment commands:
Commands in LaTeX start with the backslash symbol. eg: today, maketitle, makeindex etc.,

Environment commands start with backslash just like normal commands, but they occur in pairs. Environment commands create a region of special formats or special features within the document. They start with a begin and end with an end.
eg: begin document - end document, 

\end{document}