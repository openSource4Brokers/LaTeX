\documentclass[]{report}
 
% https://ctan.org/pkg/xecjk
\usepackage{xeCJK}
\usepackage{mathtools}
\usepackage{lipsum}
\usepackage{array}
\usepackage{colortbl}
\usepackage{chemfig}

 

\begin{document}

\tableofcontents
\listoffigures
\listoftables

\chapter{Introduction}

This is my first LaTeX document.

这是我的第一个LaTeX文档。

\section{Why LaTeX?}

白色的半圆大厅中一片死寂,仿佛时间也被封闭在这里不再流动,真的很像坟墓。以后这儿就是她的全部世界了。她首先要做的是让这里恢复生活的气息。她不想像罗辑那样,她不是战士和决斗者,她是女人,毕竟要在这里度过很长时间,可能是十年、半个世纪,其实她为这个使命准备了一生,所以站在这漫长道路的起点,她很坦然。

但命运却再次显示了它的怪异无常,程心准备了一生的执剑人生涯,从她接过红色开关时起,仅仅持续了十五分钟便结束了。

\subsection{Typesetting Signs}

This is how to insert a \% sign.

This is how to insert a \$ sign.

\subsection{Dummy Text}

\lipsum

\chapter{Mathematics}

Text mode

$Math mode$

\section{One VS Two \$ Signs}

Protons and electrons have charges of equal magnitude, but opposite sign. The magnitude of these charges is known as the elementary charge and has a value of $e=1.602\times 10^{-19}\ C$.

Protons and electrons have charges of equal magnitude, but opposite sign. The magnitude of these charges is known as the elementary charge and has a value of $$e=1.602\times 10^{-19}\ C$$

$$\frac{numerator}{denominator}$$

\section{Equation Environment}

Equation \ref{eq:Straight Line} is the equation of a straight line.
\begin{equation} \label{eq:Straight Line}
y=mx+c
\end{equation}

Equation \ref{eq:Quadratic Equation Roots} determines the roots of a quadratic equation.
\begin{equation} \label{eq:Quadratic Equation Roots}
x=\frac{-b\pm \sqrt{b^2-4ac}}{2a}
\end{equation}

\section{Long Equations}

\begin{equation}
\left[ \left( \overline{h}_f \right)_{T,P} \right]_{H_2} = \left[ \left( \overline{h}_f \right)_{T_{Ref}} \right]_{H_2} + \\
\left[M \left\{ C_0T + \frac{C_1T^2}{2(1000)} + \frac{C_2T^3}{3(1000)^2} + \frac{C_3T^4}{4(1000)^3} - \right.\right. \\
\left.\left. C_0T_{Ref} - \frac{C_1T_{Ref}^2}{2(1000)} - \frac{C_2T_{Ref}^3}{3(1000)^2} - \frac{C_3T_{Ref}^4}{4(1000)^3} \right\} \right]_{H_2}
\end{equation}

\begin{multline}
\left[ \left( \overline{h}_f \right)_{T,P} \right]_{H_2} = \left[ \left( \overline{h}_f \right)_{T_{Ref}} \right]_{H_2} + \\
\left[M \left\{ C_0T + \frac{C_1T^2}{2(1000)} + \frac{C_2T^3}{3(1000)^2} + \frac{C_3T^4}{4(1000)^3} - \right.\right. \\
\left.\left. C_0T_{Ref} - \frac{C_1T_{Ref}^2}{2(1000)} - \frac{C_2T_{Ref}^3}{3(1000)^2} - \frac{C_3T_{Ref}^4}{4(1000)^3} \right\} \right]_{H_2}
\end{multline}

\section{Aligned Equations}

\begin{align}
{x_{rms}}&=\sqrt{\dfrac{x_{amp}^2}{2\Biggl(\dfrac{2\pi}{\mu}\Biggr)}\left[ \dfrac{2\pi}{\mu}+\dfrac{\sin{\Biggl(2\mu\left(\dfrac{2\pi}{\mu}\right)+2\gamma_x\Biggr)}}{2\mu}-\dfrac{\sin{(2\gamma_x)}}{2\mu} \right]} \\
&=\sqrt{\dfrac{\mu x_{amp}^2}{4\pi}\Biggl[ \dfrac{2\pi}{\mu}+\dfrac{\sin{(4\pi +2\gamma_x)}}{2\mu}-\dfrac{\sin{(2\gamma_x)}}{2\mu} \Biggr]} \\
&=\sqrt{\dfrac{\mu x_{amp}^2}{4\pi}\Biggl[ \dfrac{2\pi}{\mu}+\dfrac{\sin{(2\gamma_x)}}{2\mu}-\dfrac{\sin{(2\gamma_x)}}{2\mu} \Biggr]} \\
&=\sqrt{\dfrac{x_{amp}^2}{2}} \\
&=\dfrac{x_{amp}}{\sqrt{2}}
\end{align}

\begin{align*}
{x_{rms}}&=\sqrt{\dfrac{x_{amp}^2}{2\Biggl(\dfrac{2\pi}{\mu}\Biggr)}\left[ \dfrac{2\pi}{\mu}+\dfrac{\sin{\Biggl(2\mu\left(\dfrac{2\pi}{\mu}\right)+2\gamma_x\Biggr)}}{2\mu}-\dfrac{\sin{(2\gamma_x)}}{2\mu} \right]} \\
&=\sqrt{\dfrac{\mu x_{amp}^2}{4\pi}\Biggl[ \dfrac{2\pi}{\mu}+\dfrac{\sin{(4\pi +2\gamma_x)}}{2\mu}-\dfrac{\sin{(2\gamma_x)}}{2\mu} \Biggr]} \\
&=\sqrt{\dfrac{\mu x_{amp}^2}{4\pi}\Biggl[ \dfrac{2\pi}{\mu}+\dfrac{\sin{(2\gamma_x)}}{2\mu}-\dfrac{\sin{(2\gamma_x)}}{2\mu} \Biggr]} \\
&=\sqrt{\dfrac{x_{amp}^2}{2}} \\
&=\dfrac{x_{amp}}{\sqrt{2}}
\end{align*}

\chapter{Figures}

\begin{figure}[!ht]
\centering
\includegraphics[width=100mm]{./Pictures/Cat.jpg}
\caption{Grumpy Cat}
\label{fig:Grumpy Cat}
\end{figure}

From figure \ref{fig:Grumpy Cat}, it is evident that the cat is grumpy.

\chapter{Tables}

Table \ref{table:water_ionization_constants} shows some ionization constants for water at different temperatures.
\begin{center}
\begin{table}[!ht]
\centering
\begin{tabular}{ !{\vrule width 1pt}c!{\vrule width 1pt}c!{\vrule width 1pt}}
\noalign{\hrule height 1pt}
\cellcolor[gray]{0.9} \textbf{Temperature $[^{\circ}\mathrm{C}]$} &
\cellcolor[gray]{0.9} \textbf{Ionization constant $K_w$}
\\ \noalign{\hrule height 1pt}
10 & $0.29\times 10^{-14}$
\\ \hline
15 & $0.45\times 10^{-14}$
\\ \hline
20 & $0.68\times 10^{-14}$
\\ \hline
25 & $1.01\times 10^{-14}$
\\ \hline
30 & $1.47\times 10^{-14}$
\\ \hline
50 & $5.48\times 10^{-14}$
\\ \noalign{\hrule height 1pt}
\end{tabular}
\caption{Ionization constants for water}
\label{table:water_ionization_constants}
\end{table}
\end{center}

\chapter{Chemistry}

\chemfig{H_3C-C(=[:90]\lewis{0:4:,O})-[:-30]\lewis{2:,N}(-[:-90]H)-[:30]C*6(-C(-[:-90]H)=C(-[:-30]H)-C(-[:30]\lewis{0:6:,O}-[:90]H)=C(-[:90]H)-C(-[:150]H)=)}

\end{document}