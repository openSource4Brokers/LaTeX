\documentclass[]{report} % 10, 11 (default) or 12
\usepackage[english]{babel}
% Instead of using \\ (or \newline/\vspace) to simulate paragraph breaks, or to increase inter-paragraph spacing,
% the recommended solution for setting spacing between paragraphs is to load the parskip package:
% put \usepackage{parskip} in your document’s preamble.
% https://ctan.org/pkg/parskip
\usepackage{lipsum}

\usepackage{array}
\usepackage{colortbl}
\usepackage[utf8]{inputenc}

\setlength{\arrayrulewidth}{1mm}
\setlength{\tabcolsep}{18pt}
\renewcommand{\arraystretch}{1.5}

\begin{document}
\tableofcontents

%%%
%%%
%%%
%PCWsBAS.WS4
%-----------
%
%- "PCW: Streamlined BASIC"
%   Geoffrey Childs
%  PCW-World, 1990
%
%A comprehensive guide to streamlining Mallard BASIC on the Amstrad PCW range
%
%(Retyped by Emmanuel ROCHE.)
%
%
%COPYRIGHT (C)
%
%Published  by  PCW-World, Meadway Court, Bloomfield Street  North,  HALESOWEN, 
%West Midlands B63 3RE, England.
%
%Copyright 1989 (c) PCW-World All rights reserved. No part of this  publication 
%may  be  reproduced  in  any  form  without  the  written  permission  of  the 
%publishers.
%
%ISBN 0 9515486 0 3
%
%First Published October 1989
%Second Edition July 1990
%
%A disc containing the programs printed herein is supplied with this book.
%
%No  liability whatsoever will be entertained by the publishers or  the  author 
%for  any  damages, consequential, indirect, incidental or  special,  resulting 
%from  the  use  of the material contained in this  book  or  the  accompanying 
%program disc.

%AMSTRAD is the registered trademark of AMSTRAD Consumer Electronics plc.
%CP/M Plus is the registered trademark of Digital Research Incorporated.
%Mallard  BASIC,  LOCOMOTIVE  and  LocoScript  are  registered  trademarks   of 
%Locomotive Software.
%All trademarks acknowledged.

\section{Acknowledgements}
My sincere thanks are due to my many correspondents and friends who have given 
me  ideas,  answered  my  queries and made helpful  comments  which  have  all 
contributed  to the preparation of this book. In particular, I would  like  to 
thank  David  Wilson,  who has made numerous suggestions (some  of  which  are 
printable)  and  assisted with the final editing. I am also very  grateful  to 
Gerry  Austin  of PCW-World, who has managed to publish this book  within  two 
months of our initial meeting -- incredible!

Geoffrey Childs, October 1989.

%%%
%%%
%%%
\section{Part ONE}
You  may  be a very good programmer, you may just think you are,  or  you  may 
realise  that  you  are only a beginner. You may  have  written  thousands  of 
programs...  or  just the one. It does not matter in the least. The  less  you 
know, the more you can learn from this book.

Will  this  book  help you to write spectacular  programs  with  mind-boggling 
graphics?  Or  will  it  just help you to write  a  program  for  your  weekly 
budgeting  that works efficiently? The answer lies solely in your hands.  This 
book  will  give  you the means to produce programs with  all  the  bells  and 
whistles. It will also help you to do the much simpler programming tasks well. 
What it will NOT do is to teach you how to PRINT "Hello". Nor will it give the 
syntax  for every Mallard BASIC word. You need an elementary tutorial for  the 
former, and a manual for the latter.

Part  1 is about the things that you CAN officially do with Mallard BASIC.  It 
is   hoped  that  the  discussions  will  improve  the  proficiency  of   your 
programming,  so that the simple things are done effectively. Part 2 is  about 
the things you CANNOT do with Mallard BASIC. As we shall see, there is no such 
word as CAN'T! Admittedly, we shall have to make some recourse to machine code 
in Part 2: but have no fear, you will not be asked to understand it. Any BASIC 
programmer can use code routines that have already been written.

The final section is a series of appendices. This includes a simple  extension 
to  Mallard BASIC and a few other programs that do not conveniently  fit  into 
the  text. There are plenty of short programs and routines that  are  included 
and explained in the context of the text. Each of these will be labelled  with 
a name such as C4P2 (short for Chapter 4 Program 2). These can be found on the 
program disc supplied with this book.

In  a book about computers, it is impossible to avoid 'jargon'  completely.  I 
hope that most of it will be explained as we progress. It may be helpful  just 
to  explain  a few words immediately. MEMORY can be thought of  as  the  brain 
cells of the computer. Each cell of memory (there are over 500,000 of them  in 
the  Amstrad  PCW8512) contains a number 0-255 that the  computer  'remembers' 
until  it is changed, or until the computer is switched off. CODE is  an  all-
purpose  word  that  means  symbols, letters, words,  and  language  that  the 
computer  can operate upon (given a little bit of help by what is  already  in 
its memory). The PROCESSOR is the electronic device that turns the numbers  in 
memory into simple activities that the computer can carry out. MACHINE CODE is 
the  code that the processor can understand. A programming  LANGUAGE  converts 
the code that a programmer writes into code that the computer can use. Mallard 
BASIC is called an INTERPRETER as it does that conversion, line by line,  each 
time  a  program runs. Other languages use a COMPILER which turns  a  complete 
program (SOURCE) into a new program (OBJECT). The object program is in machine 
code. We shall not be much concerned with compilers. RESERVED words are  words 
that have a special meaning to the computer in languages such as BASIC. PRINT, 
RANDOMIZE,  and SQR are examples of reserved words in BASIC. DO is a  reserved 
word in the Pascal language, but means nothing in BASIC.

If you did not receive a program disc when you bought the book, please let  us 
know at PCW-World ('Cotswold House', Cradley Heath, Warley, West Midlands  B64 
7NF) and we will provide a disc free of charge on receipt of some evidence  of 
the sale.

\chapter{BASIC matters}
Just  before  the second world war, LNER engine number 4468 achieved an all-time
speed record for a steam engine of 126 mph (203 km/h) between Grantham and
Peterborough. It could not have been  done 
without  'streamlining'  or a competent driver and fireman.  Number  4468  was 
named  "Mallard", as is the BASIC on your Amstrad PCW. Mallard BASIC has  also 
been  designed  to  be very fast and, when you program with it,  you  are  its 
driver and fireman. The 'streamlining', I hope, will be found in the following 
pages.

\section{Purpose}
This  book  is not intended to be a user manual for Mallard BASIC, or  even  a 
tutorial.  It  is intended for Amstrad PCW users who  have  experimented  with 
Mallard BASIC and attempted to write some serious, even if simple, programs of 
their  own, but would like to move on to bigger and better ones. If  you  have 
not reached this stage, I suggest you study some tutorial material first. Your 
first  choice  will,  obviously, be the official manual, and  a  book  by  Ian 
Sinclair has been generally recommended. For those preferring to learn at  the 
keyboard,  'David  Wilson  Computronics' have produced  a  comprehensive  disc 
entitled "BASIC Tutorial".

As  I  develop  my approach, I will not be afraid to PEEK and  POKE  about  in 
memory from time to time, but I will not assume that you have any knowledge of 
machine  code programming -- or even that you want to descend to  that  level! 
Very  occasionally,  a machine code sequence will be included  in  an  example 
routine, but you will not need to understand it.

My central theme will be programming style, though some might think that I  am 
the last person on Earth who ought to be writing about that (ROCHE>  Including 
me, since Geoffrey Childs *NEVER* indents his programs!). If someone wanted to 
compliment  my  programming, I would accept -- with typical  modesty  --  such 
words as "knowledgeable", "ingenious", "economical", and even "eccentric". But 
not  even my best friends -- nor, for that matter, my worst enemies  --  would 
call me a stylish programmer. However, what we are considering is your  style, 
not  mine. There are certain aspects of programming that I consider to be  bad 
programming  rather  than  bad style and, in these cases,  I  shall  be  quite 
dictatorial, telling you a few things that you must not do. Apart from that, I 
hope  that  you will accept most of what follows as suggestions,  rather  than 
instructions.  Very often, the question is one of speed of operation,  or  the 
comfort  of  the  user, which is what good footplate  crew  have  always  been 
concerned about.

Any form of programming is a skill and, however individual a skill may be, the 
best practitioners will try to learn from others. If you think that you always 
program  perfectly,  who am I to disagree? You may find some of my  ideas  and 
methods  reprehensible  --  again, I will not disagree. What I  can  claim  is 
experience -- the fact that I wrote "Lightning BASIC" proves that -- and  that 
most of my programs do work, eventually.

\section{BASIC extensions}
Mallard is a powerful and fast implementation of BASIC. It is fair to say that 
the  more I have used it, the more I like it. I did find it rather strange  at 
first,  after using Microsoft and Xtal BASICs on other machines. However,  any 
new system needs patience before it becomes familiar.

Having  said  this, there is no question that Mallard is extremely  clumsy  at 
doing  a few simple things, the most obvious of which is clearing the  screen! 
(ROCHE>  BASIC,  and CP/M, were first used on computers which had  no  screen, 
only  an  ASR-33 Teletype through which all interaction  with  BASIC/CP/M  was 
done.  So, no screen = no CLS. Anyway, who needs a CLS when pressing  [RETURN] 
the  appropriate  number of lines is enough to clear the screen?) There  is  a 
method  behind Mallard's madness. It is not that Locomotive were incapable  of 
devising a CLS which would have done the job. By using control codes,  Mallard 
files  become portable when saved in ASCII form, and can be used on  computers 
other than the Amstrad PCW. (ROCHE> I have been using Mallard-86 BASIC for MS-
DOS  for 10 years, now, and have *NEVER* used ASC files, only BAS  files  when 
transfering my old Amstrad PCW programs...)

Your  masterpiece (and mine) are probably only intended to run on the  Amstrad 
PCW. (ROCHE> I started as a COBOL programmer on IBM Mainframes. Then  ventured 
into minicomputers. Then finally used micro-computers (which went from 8  bits 
to  16  bits,  then 32 bits!). For me, Mallard BASIC is a  high-level  way  of 
porting  my  programs: they run under CP/M Plus, MS-DOS, and in  a  "DOS  Box" 
under  Windows.) It makes sense to make a few extensions to Mallard BASIC,  so 
that the simple jobs can be done quickly.

Much of my programming on the Amstrad PCW has been devoted to writing  various 
extensions  to  Mallard  BASIC, so that not only does it do  the  simple  jobs 
easily, but becomes more powerful generally.

I  will admit that I do most of my BASIC programming using  "Lightning  BASIC" 
but,  then, having written it myself, I have not had to pay for it! You  would 
not  like it if, having raided your piggybank to buy this book, I told you  to 
send  24.95 Pounds to CP Software to buy Lightning BASIC. (But, of course,  it 
is worth every penny...)

DWBAS is a much shorter extension (also written by me) -- this can be RUN  and 
LISTed  from  the program disc. You may have this for nothing. I do  not  like 
typing  in  programs with lots of figures, that is why I have  provided  DWBAS 
(and  the  other  programs  in this book) on disc. If  you  have  the  Amstrad 
PCW9512,  use  DW139  instead, this is also on the program disc.  This  is  an 
amended  version  that runs with Mallard-80 BASIC Version 1.39.  (The  Amstrad 
PCW8256  and  PCW8512 use Mallard-80 BASIC Version 1.29.) The  explanation  of 
these extensions is given in Appendix 3. The programs in the text do not  rely 
on DWBAS, but those in Appendices 6 and 7 do need DWBAS loaded first.

One  important feature of DWBAS (and Lightning BASIC) is a multiple POKE.  The 
command  POKE  50000,2,3,4 will POKE 2 into 50000, 3 into 50001,  and  4  into 
50002.  I find this so important that I shall INSIST that you have a  multiple 
POKE when we come to Part 2 of the book. If you refuse to use DWBAS,  Appendix 
4 gives a briefer program which will install this feature. Naturally, you will 
not need this if you use DWBAS.

\section{Your own style}
Part  1  of  this  book is intended to help you  develop  your  own  style  of 
programming. There are usually many ways of achieving the desired result in  a 
program.  Often,  the  programmer has to make a choice  between  economies  in 
memory  and  speed  on the one hand, and attractive  presentation  and  easily 
readable   programs  on  the  other.  Programming  style  generally   develops 
subconciously,  but it is not a bad idea to have an introspective  examination 
of  one's programs from time to time. In this section, we have  a  preliminary 
look at some of the questions that you might consider.

Writing  computer programs has parallels with writing English. It is  easy  to 
identify bad style, but good style is not so clear cut or uncontroversial.  In 
most  of  our  literary  works, we would try to  avoid  spelling  sausages  as 
'sosiges', but it would not matter if we put 'sosiges' on our weekly  shopping 
list!  In an idle quarter of an hour, last week, I used a computer program  to 
work out the Daily Telegraph Brain Teaser. It does not matter in the slightest 
how badly the program was written, provided it works!

Another parallel with English is that the style of a computer program  depends 
on the intended user. You would not write a love letter in the same style as a 
technical  report, and it is unlikely that you would write a "Space  Invaders" 
game in the same style as an accounts system.

Clearly, there are some general rules that all programmers should follow,  and 
they  can  be summarised in one sentence. Write your program in a  style  that 
will  be  suitable for the people you expect to use it.  The  balance  between 
lengthy  prompts, excessive input checking and confirmation on the  one  hand, 
and  a  fast  flowing program on the other, must be  made  with  the  expected 
experience of the average user in mind. Decorative screens and musical jingles 
(not  so  easy  on  the Amstrad PCW) must be  balanced  between  pleasure  and 
irritation  potential! It is sometimes hard to remember that the person  using 
your  program,  probably  a philistine, will not be  admiring  your  beautiful 
programming skills, but using your program as a means to an end.

Having  made these general didactic points, it is time to emphasise  the  main 
point I want to make in this section: DEVELOP YOUR OWN STYLE.

What  is right for you, may not be right for me, and vice versa. If  you  find 
that  you can write programs that work well without bothering about  structure 
and subroutines, then write that way -- at least until you find that it  might 
be wiser to modify your views. If you find that planning on paper and having a 
complete knowledge of exactly what your program will do is essential -- do  it 
that  way. Academic writers about computer programming talk some sense, but  I 
think that they talk a lot of rubbish, too!

I  am going to finish this section by asking a number of questions.  They  are 
question  which you may feel strongly about, and your answers may be  opposite 
to mine. You may feel that the answer is sometimes 'yes' and sometimes 'no'. I 
hope that some will be questions that you have not even considered. In a short 
section  like this, it is impossible to cover all the questions one might  ask 
about  style. Provided you are thinking about these propositions,  it  matters 
not  whether  you  become more pig-headed in your views, or  more  willing  to 
modify them.

Should variable names be long: "age.of.employee" or short: "a"?

Do you religiously, occasionally, or never put in REMs?

Do you leave blank lines, such as "200 :", to make your listings look pretty?

Should  documentation  be non-existent, in REMs, on-screen, or in  a  separate 
file or booklet?

Do you use "Press any key..." or a specific "Press SPACE..."?

Do you always use INPUT, even for one-letter replies?

Do you believe in right-justifying or centring screen printing?

Do you make sure the user sees your name on any programs you write?

Do you have any little habits by which a program you wrote would be identified 
as likely to be yours?

Should  you  avoid  POKEs or CALLs that make your programs  machine  or  model 
specific?

Do you use reverse video often, occasionally, or never?

Do  you  write your subroutines before the rest of your program,  or  whenever 
they crop up?

Do you think that you make too little or too much use of subroutines?

Do  you  like  or dislike programs where the menu choice  is  made  by  cursor 
control (or a mouse), rather than by pressing a key?

Is it better to keep most program lines to a single statement?

What  are your feelings about GOTO? That it should be never used. That is  has 
occasional uses. That you do not care what the academics think, and it is just 
about your favorite BASIC word!

Do you use integer variables whenever possible, occasionally, or never?

Do  you  RENUM  when a program is complete, or do you leave  numbers  as  they 
stand, so that you may be able to recall them better later?

Do you ever use DEF FN, apart from escape sequences? Or PEEK, POKE, CALL,  and 
VARPTR?

Do you use ON ERROR throughout a program, for occasional bits where a specific 
error could be expected, or never?

Do you protect your programs? If not, do you make your listings as easy (or as 
difficult!) as possible for the user to follow?

Which type of files do you use most often: sequential, random, or Jetsam?

Do  you think Jetsam is the best thing about Mallard BASIC, could be  used  on 
occasions, or is too complex to try to understand?

What  is  your attitude to a finished program that works? "I am not  going  to 
touch it again", or "I will take every opportunity to gild the lily"?

Do you consider it bad programming if the screen scrolls during operation?

Some  of the questions will not be discussed any further, but there are  other 
questions on which I do have definite views, and which we shall think about in 
more  detail.  Where  this happens, I shall be happy to  preach  to  heathens, 
doubters, or the converted!

\section{Planning}
If  you  are a competent pianist, it is possible to sit down at a  piano,  put 
your hands on the keys, have no idea what you are going to play, and improvise 
happily  and  tunefully. You cannot do this with a computer. All  programs  do 
need  an  element  of planning. At the very minimum, you must  know  what  the 
program aims to achieve.

Flowcharts  were  once one of my pet hates. Perhaps I do not  read  the  right 
books  these days, but thankfully we do not hear so much about them, now.  The 
idea of a flowchart is to put into words the various stages of a program,  and 
use  symbols and arrows to define its possible courses. For a simple  program, 
this  seems a waste of effort and, for a complex program, it is  difficult  to 
imagine  the  size  of paper that you would need! I  would  find  drawing  the 
flowchart much harder than writing the program -- which is not the purpose  of 
the  exercise! I know some people do find them helpful. If you like  them,  do 
not let me stop you using them. You are probably more methodical than I.

Some methods of planning are needed. If I am writing a Mallard BASIC  program, 
I  usually  code  at  the computer. Occasionally, I may use  the  back  of  an 
envelope  for  a section of code that looks tricky. I may write a  very  rough 
menu  outline  for  a long program on paper. If I am writing  a  machine  code 
routine, I usually write it in assembler on an envelope, and translate it into 
code at the machine. Not recommended, unless you also are in an advanced state 
of madness! The stationers do not make a fortune out of me. I appreciate  that 
most people would disagre with so little paper planning. Ask yourself if  more 
time  spent on paper planning would make the eventual writing easier  or  more 
efficient. Only you can decide.

You may have kindly thoughts like: "Yes, but you are such a genius that you do 
not  need to plan." Or unkind ones like: "No wonder your programs turn out  so 
badly." Wrong on both counts. I do plan: I probably spend more time on it that 
you do. If an idea for a substantial program comes to mind, I do not sit  down 
at  the  computer to write it. I carry the idea, maybe for months,  and  think 
about  it.  I do nothing at the computer until I know that I  have  sufficient 
time  to  write the program with a minimum of breaks. By the time I  start,  I 
have  a clear outline plan in my mind. While I am writing the program, I  will 
be  thinking about it and the little extras that can be introduced to  improve 
on  the outline. When the program is written, I hope to have nothing  more  to 
do. Usually, this is optimistic because it is often someone else who finds the 
bugs.

'Structured programming' is a computer buzz-word. To some people, it is closer 
to  a religious utterance. A structured program is contained in modules. On  a 
computer such as the BBC, these modules are called PROCEDURES. On the  Amstrad 
PCWs,  we have to make do with subroutines, which are nearly the  same  thing! 
The  procedures are written, and all the main program does is to call them  in 
the  correct sequence. My advice is not to make structured programming into  a 
religion,  but  make  good use of subroutines, so that your  programs  have  a 
semblance of structure -- even if served with a helping of spaghetti!

If I use any formal technique, it is probably closer to the 'top-down'  method 
than anything else. I usually a program at line 5000, where I put the  general 
purpose subroutines -- some will have been pinched from other programs. I  may 
write  some more specific subroutines, such as filing ones, starting  at  line 
6000, which may only need to be chained to certain options on the menu. I  use 
7000+ for error traps. The main program is divided into sections, starting  at 
line  1000,  2000,  etc. I write the program a section at  a  time,  when  the 
subroutines  have  been  written.  Sometimes, I also  use  lines  above  8000. 
Eventually,  the  main  menu and housekeeping goes at  the  beginning  of  the 
program.

There is no magic in this particular numbering system, but there is plenty  of 
point  in establishing a pattern for yourself and sticking to it, if only  for 
the fact that anything done consistently is more easily remembered.

However  well  you  plan, you will probably spend at least  as  much  time  on 
testing  and debugging your program as on writing it, but we will leave  these 
little treasures until later.

%%%
%%%
%%%
\chapter{Everyday problems}
In this chapter, we are going to look at some Mallard BASIC commands that most 
of  you  will  have  used. Familiarity can breed contempt,  and  good  or  bad 
programming is often more evident in the simple everyday operations that  most 
programmers will be using frequently. The topics may seem disjointed, but  the 
common  theme is that we should think about the routines that we know  we  can 
program, as well as those that we know will offer a new challenge.

\section{Escape sequences}
One  of  the first things most of use notice about Mallard BASIC is  that  you 
cannot clear the screen with a simple CLS (or something similar). I have  come 
to  respect Mallard in many ways, but I still think that this  is  ridiculous, 
even though the reason is that Mallard BASIC is meant to be portable.

If  you are using DWBAS, you will be happy not having to use escape  sequences 
like

        $$PRINT CHR\$ (27) + "E" + CHR\$ (27) + "H"$$

any more.

This  will be the case if you only want to program for yourself. Some  of  you 
will wish to write programs for magazines, for example, and you cannot rely on 
readers having DWBAS. Life will be simpler with DWBAS, but there will still be 
times when you have to use the conventional sequences.

So,  let  me get rid of a couple of bees in my bonnet. I hate the  people  who 
write

        $$esc\$ = CHR\$ (27)$$

or even worse:

        $$escape\$ = CHR\$ (27)$$

The idea is, presumably, to shorten things, so WHY NOT

        $$e\$ = CHR\$ (27) ?$$

It  also irritates me greatly to see the PRINT AT sequence used without a  DEF 
FN, and the horrible CHR\$(32+r) appearing all over a program. If you must  use 
escape sequences, my advice is to write a little subprogram including all  the 
regular ones, and start each new program with it. Unused escape sequences  can 
be deleted when the program is completed. (End of burst of bad temper!)

It is worth studying pages 140-141 (in my copy) of Manual 1 to note the  rarer 
escape  sequences. Some knowledge of the ones to use with the printer is  also 
advisable.  The effect of escape sequences is, of course, quite  different  in 
PRINT statements from that in LPRINT statements.

\section{PRINT and LPRINT}
Many  programs will contain sections that offer the user a choice of  printing 
on  the screen, or sending the same text to the printer for hard  copy.  Quite 
often,  these  program  sections are fairly long, and it seems  to  be  wasted 
effort  when the whole thing has to be written out again with all  the  PRINTs 
changed  to LPRINTs. Of course, it is possible to save some of the  burden  by 
suitable use of AUTO and RENUM, but there is a much easier way.

The idea is very simple. We temporarily send all PRINT commands to the address 
for  LPRINT commands. When we have done the LPRINTing, we change  the  address 
back to normal. POKE 18527,90 changes PRINT to LPRINT. POKE 18527,100  reverts 
to  normal.  On the Amstrad PCW9512, a different version of  Mallard-80  BASIC 
(Version 1.39, as opposed to Version 1.29) is provided, and the addresses  are 
different. The corresponding POKEs are POKE 18591,0 and POKE 18591,10. If  you 
are  writing  for users who might be in either version of Mallard  BASIC,  you 
will have to test. The address I use (out of many possibles) is 5103. If  PEEK 
(5103) = 197, then you are using Mallard-80 BASIC Version 1.39.

There is also a simple POKE to make the PRINTing go to the screen and then  to 
the  printer. POKE 8792,205 in Version 1.29 and POKE 29161,205 in  Ver.  1.39. 
This  make  a machine code jump into a call (like a GOSUB): when it  does  the 
PRINT routine it returns, and where has it got to? The LPRINT routine!  Making 
this change is rather more comprehensive than the previous POKE. Not only does 
PRINT get echoed, but also everything else that goes to the screen,  including 
the  "Ok" prompt! To go back to normal, you POKE 8792,195 with Ver. 1.29,  and 
POKE 29161,195 with Ver. 1.39.

An  omission in Mallard is that there is no official way to TYPE or DISPLAY  a 
file to hard copy. Byt try this in Mallard 1.29:

        POKE 8793, 234 : DISPLAY "FILENAME.TYP" : POKE 8793, 239

Change  8793 to 29162 with Ver. 1.39. If you run an Amstrad PCW9512, you  will 
probably  be using a version of CP/M Plus named 'J21CPM3.EMS' (found  on  your 
Amstrad  master  disc).  In  this case, you should replace  the  234  and  239 
mentioned above with 246 and 251, respectively.


%%%
%%%
%%%

\listoftables
\listoffigures


Appendices
----------

1: Program disc
2: Turnkey discs
3: Documentation for DWBAS
4: Multiple POKE
5: Disassembler
6: Menu subroutine
7: Multiple input
8: Bibliography and discotheque
9: Printer fonts
10: LNER 'Mallard' listing
11: Notes to the second edition


Index

\end{document}


